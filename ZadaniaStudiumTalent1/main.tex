\documentclass{article}
\usepackage[utf8]{inputenc}

\title{Zadania Studium Talent 1}
\author{Mikołaj Mężyk }
\date{December 2021}

\begin{document}

\maketitle

\section{Definicja kresu dolnego}
\Large
L = inf B \Leftrightarrow \begin{array}{cc}
     &  1. \forall $ $ b \in B : b \geq L\\
     & 2. \forall $ $ \varepsilon > 0 $ $\exists b_\varepsilon \in B < L- \varepsilon
\end{array}
\section{Ciągi}
Niech m = Cn\\
Jeśli n\to \infty,$ to m$\to\infty\\
\lim_{m\to\infty}(1+\frac{1}{m})^m=e\\
\\
$Jeśli Cn$\to -\infty $, to $ (1+\frac{1}{Cn})^{Cn}_{n\to\infty}\to e\\
\lim_{n\to\infty}(1+\frac{1}{Cn})^{Cn}=L\\
$Niech m = -Cn$ \longrightarrow$jeżeli n$\to\infty$, to m$\to\infty\\
L = \lim_{m\to\infty}(1-\frac{1}{m})^{-m}=\lim_{m\to\infty}(\frac{1}{\frac{m-1}{m}})^m=\lim_{m\to\infty}(\frac{m}{m-1}})^m\\
$Niech a = m-1$\\
L = \lim_{a\to\infty}(\frac{a+1}{a})^{a+1}=\lim_{a\to\infty}[(\frac{a+1}{a})(1+\frac{1}{a})^a]=e \\
\lim_{n\to\infty}(\frac{a+1}{a})=1 $ $\land$ $\lim_{n\to\infty}(1+\frac{1}{a})^a=e
\section{Limesy}
\Large
\lim_{n \to \infty} (\frac{5n-1}{5n+3})^{2n-1} \\
\\
(\frac{5n-1}{5n+3})^{2n-1} = [(\frac{5n+3}{5n-1})^{2n-1}]^{-1} = [(\frac{5n-1}{5n-1} + \frac{4}{5n-1})^{2n-1}]^{-1} = \\
\\
= [(1 + \frac{1}{\frac{5n-1}{4}})^{\frac{5n-1}{4}*\frac{8}{5} - \frac{3}{5}}]^{-1} = 
[e^{\frac{8}{5}}:1]^{-1} = e^{-\frac{8}{5}}
\\
\\
\\
\lim_{n \to \infty} (\frac{5n-1}{6n+2})^n = 0, $ ponieważ $ \lim_{n \to \infty} (\frac{5n-1}{6n+2}) = 0
\\
\\
\lim_{n \to \infty} (\frac{5n-1}{3n+2})^n = \infty, $ ponieważ $ \lim_{n \to \infty} (\frac{5n-1}{3n+2}) = \infty

\end{document}
